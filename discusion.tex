\documentclass{article}
\author{Luciano Moffatt}
\usepackage[utf8]{inputenc}
\usepackage{amsmath}
\usepackage{hyperref}
\begin{document}

\title{semsys}

semsys es un semantic system. La idea es que sea una template library que maneje magnitudes (quantities), unidades y que sea la base para armar sistemas de ecuaciones. 

En principio tendria que armar el sistema de unidades mks y cgs. 
Y lograr escribir y leer mediciones en dicho sistema. 
Tendria que sumar y restar mediciones y multiplicar y dividir mediciones. 

Luego tendria que incorporar vectores y matrices de mediciones. 

Estos vectores y matrices tendrian indices semantizados tambien. 

Ahora, hay dos formas de semantizar. 
Por magnitud, la cual es ineludible y por identidad, la cual esta determinada por posicion y naturaleza. 

Hay que distinguir objetos y eventos. 

Los objetos definen sistema de referencia espacial y los eventos sistema de referencia temporal. 
(quizas los eventos definan ambos..)


Bueno, lista de quantities
\begin{enumerate}
\item lenght meter
\item time second
\item mass kilogram
\item electric current Ampere
\item thermodynamic temperature Kelvin
\item amount of sustance mole
\item luminous intensity candela
\end{enumerate}


Bueno, estoy tratando de implementar el manejo de quantities y unidades. 

Pregunta numero uno: 
diferencio entre quantities y unidades?
Ejemplo, velocidad. Es una magnitud, pero no tiene unidad, es una unidad derivada. 
Entonces, cuando señalo una variable, tengo que decir las unidades. 


\section{viernes 24 de noviembre}
Avance bastante con semsys. 
Veo claramente el problema de la semantica: es decir la asignacion de nombres a objetos matematicos, es decir elementos de la categoria. 
Estos elementos de la categoria se los puede generar a partir de elementos "basicos". Pensando en el grupo de las palabras conmutativas... 
Entonces, la idea es que la clase grupo sea la que contiene las definiciones y los indices. 


Finalmente logre que corra la 








\end{document}